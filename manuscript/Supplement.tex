\documentclass[10pt,onecolumn,twoside]{optica-suppl-materials}
\setboolean{shortarticle}{false}

\title{Abundance of ice-associated seals in the eastern Bering Sea in spring of 2012 and 2013: supplementary material}

\author[1,*]{Paul B. Conn}
\author[1]{Erin E. Moreland}
\author[1]{Jay M. Ver Hoef}
\author[1]{Brett T. McClintock}
\author[1]{Josh M. London}
\author[1]{Michael F. Cameron}
\author[1]{Peter L. Boveng}

\affil[1]{NOAA-NMFS, Alaska Fisheries Science Center, National Marine Mammal Laboratory, 7600 Sand Point Way NE, Seattle, WA, 98115, U.S.A.}

\affil[*]{Corresponding author: paul.conn@noaa.gov}

% To be edited by editor
% \dates{Compiled \today}

% To be edited by editor
% \doi{\url{http://dx.doi.org/10.1364/optica.99.099999.s001} [supplementary document doi]}

\begin{abstract}
In this supplement, we give details on variance-covariance calculations and provide covariate plots and goodness-of-fit diagnostics for Tweedie models fitted to count data.
\end{abstract}

\setboolean{displaycopyright}{false} %copyright not used in supplementary materials

\begin{document}

\maketitle

%\section{Introduction}

%\section{Figures and Tables}
\section{Methods to compute the variance-covariance matrix of stage-class-averaged predictions}

In the main article text, we described how mean haul-out predictions were averaged over age classes using stable stage proportions as
weights.  In particular, letting $j$ index the spatial unit and day surveyed, $j \in \{1,2,\hdots,J\}$, such that
\begin{equation*}
  a_{j,k} = \sum_c \pi_{k,c} \tilde{a}_{j,k,c},
\end{equation*}
where $\tilde{a}_{j,k,c}$ denotes an availability prediction for sex-age class $c$ ($c \in 1,2,\hdots,C$) and species $k$ at spatio-temporal index $j$, and $\pi_{k,c}$ gives one of the $C$ stable stage proportions estimated by London et al. \cite{LondonEtAl2018} for species $k$.  In our example, we include $C=4$ stage classes, corresponding to $c=1$: young-of year; $c=2$: subadult; $c=3$: adult female; $c=4$: adult male.

\hspace{.5in} To compute the variance-covariance matrix associated with these weighted predictions, it is useful to write predictions in matrix form.  In particular, we write
\begin{equation*}
  {\bf a}_{k} = \textbf{M}_k \tilde{\bf a}_k,
\end{equation*}
where ${\bf a}_{k} $ is a length $J$ column vector of stage-averaged availability probabilities for species $k$, $\tilde{\bf a}_k$ is a ($J \times C$) column vector of stage class-specific predictions.  We structure $\tilde{\bf a}_{k} $ as $[\tilde{\bf a}_{1,k,1},\tilde{\bf a}_{1,k,2},\tilde{\bf a}_{1,k,3},\tilde{\bf a}_{1,k,4},\tilde{\bf a}_{2,k,1},\hdots ]$.  We then structure $\textbf{M}_k$ as
\begin{equation*}
 \textbf{M}_k = {\bf I} \otimes \boldsymbol{\pi}_k,
\end{equation*}
where ${\bf I}$ is a $(J \times J)$ identity matrix, and
\begin{equation*}
  \boldsymbol{\pi}_k = \begin{bmatrix} \pi_{k,1} & \pi_{k,2} & \pi_{k,3} & \pi_{k,4}  \end{bmatrix}
\end{equation*}
The variance-covariance matrix of predictions $\tilde{\bf a}$, $\tilde{\boldsymbol{\Sigma}}_k$ is available from the glmmLDTS R package \citep{VerHoefEtAl2010}.  We can then use the well known identity
\begin{equation*}
  \hat{\boldsymbol{\Sigma}_k} = \textrm{Var}({\bf M}_k \tilde{\bf a}_k ) = {\bf M}_k \tilde{\boldsymbol{\Sigma}}_k {\bf M}_k^\prime
\end{equation*}
to compute the variance-covariance matrix for time-averaged predictions.

\vspace{1cm}
\section{Methods to incorporate uncertainty about classification probabilities}

McClintock et al. \cite{McClintockEtAl2015} conducted a Bayesian analysis of seal classification probabilities using data from
a multiple observer experiment.  In particular, they produced posterior estimates of $\psi_{ko}$, the probability that species $k$ ($k \in \{1,2,3,4\}$) will be assigned to one of 13 different observation types ($o \in \{1,2,\hdots,13\}$).  The 13 observation types included all factorial combinations of the 4 seal species and 3 levels of uncertainty (certain, likely, or guess), together with a 13th type corresponding to ``unknown species."  However, in order to make parameters identifiable, the analysis required that analysts correctly identify all ``certain" observations (i.e., for instance, the probability of classifying a spotted seal seal as ``ribbon-certain," ``bearded-certain," or ``ringed-certain" were all set to 0).  There were thus $4 \times 13 - 4 \times 3 = 40$ species classification probabilities that were not set to zero.  However, there is also a sum to one constraint, such that $\sum_o \psi_{ko} = 1.0$.  Thus there are 36 effective parameters in the species classification model (9 for each species).

To incorporate uncertainty in these parameters for each species, we first transformed posterior samples of species classification probabilities to the multinomial logit scale.  To do so, we use the following algorithm for each species:
\begin{enumerate}
  \item Establish the ``certain" category for each species (say $o(k)$) as the baseline, setting $\beta_{o(k)} = 0.0$,
  \item Transform other posterior samples as $\tilde{{\beta}}_{ko} = \log(\psi_{ko}) - \log(\psi_{o(k)})$.
\end{enumerate}
We were then able to use the posterior samples $\tilde{\boldsymbol{\beta}}_{ko}$ to estimate an empirical multivariate normal prior distribution for $\boldsymbol{\beta}_k$ (a column vector of 9 values for each species), such that
\begin{equation*}
  \boldsymbol{\beta}_k \sim \textrm{Multivariate normal} ( \boldsymbol{\mu}_{\psi,k} , \boldsymbol{\Sigma}_{\psi,k}),
\end{equation*}
where $\boldsymbol{\mu}_{\psi,k}$ represents the mean of posterior samples $\tilde{\boldsymbol{\beta}}_{ko}$, and $\boldsymbol{\Sigma}_{\psi,k}$ is the associated variance-covariance matrix.  Translation back to the real scale was accomplished through the inverse of the multinomial logit link (i.e. $\psi_{k,o} = \exp(\beta_{ko}) \left( \sum_o \exp(\beta_{ko}) \right) ^ {-1}$).

\section{Goodness-of-fit}

We used randomized quantile residuals \cite{DunnSmyth1996} to assess whether the Tweedie models with day effects fit our count data.  Lack of fit can be diagnosed by inspecting whether such residuals differ from a uniform distribution.  Histograms of these residuals did not suggest any substantial departures from uniformity, and $\chi^2$ tests for each of the observation classes (with 20 equally spaced bins) resulted in p-values that were mostly $>0.05$ (Figs \ref{fig:gof2012}-\ref{fig:gof2013}).  Inspection of maps of residuals did not reveal any clear patterns of spatial structuring (plots not shown).  As such, we conclude that the Tweedie models result in reasonable fits to seal count data.


\begin{figure}[htbp]
\centering
\fbox{\includegraphics[width=\linewidth]{GOF2012}}
\caption{A histogram of randomized quantile residuals \cite{DunnSmyth1996} for the Tweedie model with estimated day effects on detection probability fitted to 2012 aerial seal counts.}
\label{fig:gof2012}
\end{figure}

\begin{figure}[htbp]
\centering
\fbox{\includegraphics[width=\linewidth]{GOF2013}}
\caption{A histogram of randomized quantile residuals \cite{DunnSmyth1996} for the Tweedie model with estimated day effects on detection probability fitted to 2012 aerial seal counts.}
\label{fig:gof2013}
\end{figure}

\section{Covariate effects}

For completeness, we plot marginal effects of each covariate on relative probabilities of species presence (Fig. \ref{fig:covs}).  Some of these effects (such as ice concentration) may be interpreted more or less at face value, but the reader should keep in mind that covariate effects are estimated within a multiple regression framework, and that many of the variables are autocorrelated (e.g., distance from shelf break tends to be negatively related to distance from land, depth, and positively related to the distance from the southern ice edge).  As such, naive interpretation is not likely to be very useful.
%\section*{References}

\begin{figure}[ht]
\centering
\includegraphics[width=\linewidth]{Cov_eff_plots}
\caption{Marginal effect of habitat covariates on relative probabilities of seal occurrence by species and year. Probabilities were calculated for each of 100 equally spaced locations within the observed covariate space for each habitat variable.}
\label{fig:covs}
\end{figure}

% Bibliography
\bibliography{master_bib}

%Manual citation list
%\begin{thebibliography}{1}
%\bibitem{Zhang:14}
%Y.~Zhang, S.~Qiao, L.~Sun, Q.~W. Shi, W.~Huang, %L.~Li, and Z.~Yang,
 % \enquote{Photoinduced active terahertz metamaterials with nanostructured
  %vanadium dioxide film deposited by sol-gel method,} Opt. Express \textbf{22},
  %11070--11078 (2014).
%\end{thebibliography}

\end{document} 